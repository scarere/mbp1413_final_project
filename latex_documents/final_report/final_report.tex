\documentclass[journal, final]{IEEEtran}

% Title, authors info etc.
\title{Navigating Data Augmentations for Segmentation}
\author{
    Shawn Carere\textsuperscript{$\ast \dagger \star$}, 
    Sayan Nag\textsuperscript{$\ast \dagger \diamond$}
    \thanks{Both authors contributed equally to this work}
    \thanks{\textsuperscript{$\ast$} Department of Medical Biophysics, University of Toronto, Canada}
    \thanks{\textsuperscript{$\dagger$} Techna Institute, Toronto, Canada}
    \thanks{
        \textsuperscript{$\star$} shawncarere@gmail.ca, 
        \textsuperscript{$\diamond$} sayan.nag@mail.utoronto.ca
        }
    \thanks{Last edited on \today}
    }


\begin{document}
    \maketitle
    \markboth{Biomedical Applications of AI, February 2022}{}
    \IEEEpubid{MBP1413~Winter 2022}

    \begin{abstract}
        This is a placeholder abstract which will describe the premise
        of our paper
    \end{abstract}

    \begin{IEEEkeywords}
        data augmentation, nuclei segmentation, deep learning, U-Net
    \end{IEEEkeywords}

    \section{Introduction}\label{s:intro}
        \IEEEPARstart{T}{his} is the beginning of the introduction. Here
        we will motivate why we want to explore the use of data augmentation,
        particularly for images, in the field of machine learning. 
        
        Furthermore
        we will introduce nuclei segmentation as the task for which we will
        be testing data augmentations. We will explain why nuclei
        segmentation is an important problem in general, why we chose this
        task to test data augmentation, why data augmentation is important
        for segmentation problems etc.
    \section{Background}\label{s:background}
        In this section we will describe the field and existing work done
        for nuclei segmentation
    \section{Methods}\label{s:methods}
        \subsection{Dataset}\label{ss:dataset}
            Here we will describe the dataset we chose to use. Test TRAIN
            split. Downsampling etc.
        \subsection{Models}\label{ss:models}
            Here we will describe the different models that we tried.
            We could give their baseline results without data aug here as
            well.
        \subsection{Experimental Design}\label{ss:experiments}
            Here we will explain which models we decided to run
            data aug experiments on and why. We will also describe the
            different data aug experiments we chose to Run
            \subsubsection{Experiment 1}
                Just realized that subsubsections in this template show
                up as more of a list than a section
            \subsubsection{Experiment 2}
                We will need to keep the experiment descriptions brief then
            \subsubsection{Experiment 3}
                Might need to reformat this section bc not sure how i feel
                about the subsubsections
    \section{Results}\label{s:results}
        \subsection{Results from Experiment 1}
        \subsection{Results from Experiment 2}
        \subsection{Results from Experiment 3}

    \section{Discussion}
        I've always thought of the discussion as sort of a conclusion of
        for the results  section
    \section{Conclusion}
        Briefly summaraize the keypoints from the paper
    \section*{Awknowledgments}
        Thank the course instructors as well as PI's that took the time to give
        lectures/presentationss
    \appendix
        Just in case we need one. If we have more than 1 distinct idea
        and or figure that needs to go here, then it should be changed to
        $\backslash${\tt{appendices}}, and we will create sub appendices
        \section*{References}
\end{document}